%%%%%%%%%%%%%%%%%%%%%%%%%%%%%%%%%%%%%%%%%
% Wenneker Article
% LaTeX Template
% Version 2.0 (28/2/17)
%
% This template was downloaded from:
% http://www.LaTeXTemplates.com
%
% Authors:
% Vel (vel@LaTeXTemplates.com)
% Frits Wenneker
%
% License:
% CC BY-NC-SA 3.0 (http://creativecommons.org/licenses/by-nc-sa/3.0/)
%
%%%%%%%%%%%%%%%%%%%%%%%%%%%%%%%%%%%%%%%%%

%----------------------------------------------------------------------------------------
%	PACKAGES AND OTHER DOCUMENT CONFIGURATIONS
%----------------------------------------------------------------------------------------

\documentclass[10pt, a4paper, twocolumn]{article} % 10pt font size (11 and 12 also possible), A4 paper (letterpaper for US letter) and two column layout (remove for one column)

%%%%%%%%%%%%%%%%%%%%%%%%%%%%%%%%%%%%%%%%%
% Wenneker Article
% Structure Specification File
% Version 1.0 (28/2/17)
%
% This file originates from:
% http://www.LaTeXTemplates.com
%
% Authors:
% Frits Wenneker
% Vel (vel@LaTeXTemplates.com)
%
% License:
% CC BY-NC-SA 3.0 (http://creativecommons.org/licenses/by-nc-sa/3.0/)
%
%%%%%%%%%%%%%%%%%%%%%%%%%%%%%%%%%%%%%%%%%

%----------------------------------------------------------------------------------------
%	PACKAGES AND OTHER DOCUMENT CONFIGURATIONS
%----------------------------------------------------------------------------------------

\usepackage[english]{babel} % English language hyphenation

\usepackage{microtype} % Better typography

\usepackage{amsmath,amsfonts,amsthm} % Math packages for equations

\usepackage[svgnames]{xcolor} % Enabling colors by their 'svgnames'

\usepackage[hang, small, labelfont=bf, up, textfont=it]{caption} % Custom captions under/above tables and figures

\usepackage{booktabs} % Horizontal rules in tables

\usepackage{lastpage} % Used to determine the number of pages in the document (for "Page X of Total")

\usepackage{graphicx} % Required for adding images

\usepackage{placeins} % For FloatBarrier

\usepackage{enumitem} % Required for customising lists
\setlist{noitemsep} % Remove spacing between bullet/numbered list elements

\usepackage{sectsty} % Enables custom section titles
\allsectionsfont{\usefont{OT1}{phv}{b}{n}} % Change the font of all section commands (Helvetica)

%----------------------------------------------------------------------------------------
%	MARGINS AND SPACING
%----------------------------------------------------------------------------------------

\usepackage{geometry} % Required for adjusting page dimensions

\geometry{
	top=1cm, % Top margin
	bottom=1.5cm, % Bottom margin
	left=2cm, % Left margin
	right=2cm, % Right margin
	includehead, % Include space for a header
	includefoot, % Include space for a footer
	%showframe, % Uncomment to show how the type block is set on the page
}

\setlength{\columnsep}{7mm} % Column separation width

%----------------------------------------------------------------------------------------
%	FONTS
%----------------------------------------------------------------------------------------

\usepackage[T1]{fontenc} % Output font encoding for international characters
\usepackage[utf8]{inputenc} % Required for inputting international characters

\usepackage{XCharter} % Use the XCharter font

%----------------------------------------------------------------------------------------
%	HEADERS AND FOOTERS
%----------------------------------------------------------------------------------------

\usepackage{fancyhdr} % Needed to define custom headers/footers
\pagestyle{fancy} % Enables the custom headers/footers

\renewcommand{\headrulewidth}{0.0pt} % No header rule
\renewcommand{\footrulewidth}{0.4pt} % Thin footer rule

\renewcommand{\sectionmark}[1]{\markboth{#1}{}} % Removes the section number from the header when \leftmark is used

%\nouppercase\leftmark % Add this to one of the lines below if you want a section title in the header/footer

% Headers
\lhead{} % Left header
\chead{\textit{\thetitle}} % Center header - currently printing the article title
\rhead{} % Right header

% Footers
\lfoot{} % Left footer
\cfoot{} % Center footer
\rfoot{\footnotesize Page \thepage\ of \pageref{LastPage}} % Right footer, "Page 1 of 2"

\fancypagestyle{firstpage}{ % Page style for the first page with the title
	\fancyhf{}
	\renewcommand{\footrulewidth}{0pt} % Suppress footer rule
}

%----------------------------------------------------------------------------------------
%	TITLE SECTION
%----------------------------------------------------------------------------------------

\newcommand{\authorstyle}[1]{{\large\usefont{OT1}{phv}{b}{n}\color{DarkRed}#1}} % Authors style (Helvetica)

\newcommand{\institution}[1]{{\footnotesize\usefont{OT1}{phv}{m}{sl}\color{Black}#1}} % Institutions style (Helvetica)

\usepackage{titling} % Allows custom title configuration

\newcommand{\HorRule}{\color{DarkGoldenrod}\rule{\linewidth}{1pt}} % Defines the gold horizontal rule around the title

\pretitle{
	\vspace{-30pt} % Move the entire title section up
	\HorRule\vspace{10pt} % Horizontal rule before the title
	\fontsize{20}{24}\usefont{OT1}{phv}{b}{n}\selectfont % Helvetica
	\color{DarkRed} % Text colour for the title and author(s)
}

\posttitle{\par\vskip 15pt} % Whitespace under the title

\preauthor{} % Anything that will appear before \author is printed

\postauthor{ % Anything that will appear after \author is printed
	\vspace{10pt} % Space before the rule
	\par\HorRule % Horizontal rule after the title
	\vspace{20pt} % Space after the title section
}

%----------------------------------------------------------------------------------------
%	ABSTRACT
%----------------------------------------------------------------------------------------

\usepackage{lettrine} % Package to accentuate the first letter of the text (lettrine)
\usepackage{fix-cm}	% Fixes the height of the lettrine

\newcommand{\initial}[1]{ % Defines the command and style for the lettrine
	\lettrine[lines=3,findent=4pt,nindent=0pt]{% Lettrine takes up 3 lines, the text to the right of it is indented 4pt and further indenting of lines 2+ is stopped
		\color{DarkGoldenrod}% Lettrine colour
		{#1}% The letter
	}{}%
}

\usepackage{xstring} % Required for string manipulation

\newcommand{\lettrineabstract}[1]{
	\StrLeft{#1}{1}[\firstletter] % Capture the first letter of the abstract for the lettrine
	\initial{\firstletter}\textbf{\StrGobbleLeft{#1}{1}} % Print the abstract with the first letter as a lettrine and the rest in bold
}

%----------------------------------------------------------------------------------------
%	BIBLIOGRAPHY
%----------------------------------------------------------------------------------------

\usepackage[backend=bibtex,style=authoryear,natbib=true]{biblatex} % Use the bibtex backend with the authoryear citation style (which resembles APA)

\addbibresource{example.bib} % The filename of the bibliography

\usepackage[autostyle=true]{csquotes} % Required to generate language-dependent quotes in the bibliography
 % Specifies the document structure and loads requires packages

%----------------------------------------------------------------------------------------
%	ARTICLE INFORMATION
%----------------------------------------------------------------------------------------

\title{Sollten Algorithmen im Gericht benutzt werden, um die Rückfälligkeitswahrscheinlichkeit von Angeklagten zu berechnen? } % The article title

\author{
	\authorstyle{Gruppe: 10\\ Journalist: Jonas Opitz\\ Chefredakteur: Frank Eric Mbouga} % Authors
}

% Example of a one line author/institution relationship
%\author{\newauthor{John Marston} \newinstitution{Universidad Nacional Autónoma de México, Mexico City, Mexico}}

\date{\today} % Add a date here if you would like one to appear underneath the title block, use \today for the current date, leave empty for no date

%----------------------------------------------------------------------------------------

\begin{document}

\maketitle % Print the title

\thispagestyle{firstpage} % Apply the page style for the first page (no headers and footers)


\section{Einleitung}
Welche Folgen kann es für wen geben, wenn man einen Algorithmus zum Bestimmen der Rückfälligkeitswahrscheinlichkeit von Angeklagten, wie OCEAN, im Gericht einführt?

Nachdem es im vorherigen Blogeintrag darum ging, wie Algorithmen wie Northpointes OCEAN funktionieren, soll es nun um die Konsequenzen der Einführung dieser gehen.

Um diese Frage zu betrachten wird zunächst das soziale System, d.h. die betroffenen Akteure, betrachtet, in das diese Algorithmen eingefügt werden, und ob es sich hier nach Kienle und Kunau um ein sozio-technisches System handelt.
Danach wird noch ein Mal auf die Funktionsweise von OCEAN eingegangen und diskutiert, woher die in [1] beschriebenen rassistischen Tendenzen her kommen könnten.
Zuletzt wird die primäre These dieser Diskussion über ein Vestersches Wirkungsgefüge gekräftigt.

Da Algorithmen zum Bestimmen der Rückfälligkeitswahrscheinlichkeit bereits in den USA angewandt werden, und sich die meisten Studien auf die USA beziehen, wird sich dieser Blogeintrag ebenfalls auf die USA beziehen.

\section{Das interagierende soziale System}
Auf erstem Blick scheint es vier Akteure zu geben, die von Algorithmen wie OCEAN betroffen sind:
\begin{itemize}
  \item Die Privatunternehmen, die solche Algorithmen entwickeln, vermieten und somit von ihnen profitieren.
  \item Der Gesetzgeber, der gegebenenfalls den Einsatz dieser Algorithmen regulieren muss.
  \item Die Richterschaft, für die diese Algorithmen überhaupt gemacht werden, und deren Arbeit durch diese unterstützt werden soll.
  \item Die Angeklagten, für die diese Algorithmen entscheiden, ob sie vor der Gerichtsverhandlung in das Gefängnis gehen bzw. bail zahlen müssen.
\end{itemize}
Jedoch lässt sich die letzte Gruppe von Menschen, die Angeklagten, in zwei Subgruppen unterteilen: 
\begin{itemize}
  \item Die, die historisch im Gericht bevorteilt sind (d.h. weiße Amerikaner).
  \item Die, die historisch im Gericht benachteiligt sind (besonders Afro-Amerikaner).
\end{itemize}
Dass diese Unterteilung sinnvoll ist folgt aus Studien wie [4], in denen untersucht, und bestätigt [4, Kapitel 7], wurde, ob es eine Korrelation zwischen schwereren Gerichtsurteilen und der Ethnizität der/des Angeklagten gibt.

\section{Handelt es sich um ein sozio-informatisches System?}
Um ein sozio-informatisches System nach Kienle/Kanau handelt es sich hier nicht, da die dritte Bedingung, "das technische System findet Eingang in die Selbstbeschreibung des sozialen Systems" [5], nicht erfüllt ist - das Verwenden von Algorithmen wie OCEAN dient lediglich der Unterstützung von Richtern und führt zu keiner Neuheit, die in die Selbstbeschreibung des sozialen Systems einhergehen würde.

\section{Kann ein Fragebogen rassistisch sein?}


\section{Vestersches Wirkungsgefüge}


\section{Schluss}

\pagebreak
\section{Quellen}
\begin{itemize}
  \item{[1]}: 
    Julia Angwin, Jeff Larson, Surya Mattu, Lauren Kirchner (ProPublica): \textit{“Machine Bias”}, 2016\\
    https://www.propublica.org/article/machine-bias-risk-assessments-in-criminal-sentencing (abgerufen am 19.03.2019)

  \item{ [2]}: 
    Northpointe Inc.: \textit{"Practitioner’s Guide to COMPAS Core"}, 2015\\
    https://assets.documentcloud.org/documents/2840784/Practitioner-s-Guide-to-COMPAS-Core.pdf (abgerufen am 19.03.2019)

  \item  {[3]}:
    Julia Dressel, Hany Farid: \textit{"The accuracy, fairness, and limits of predicting recidivism"},
    Publiziert 2018 in \textit{Science Advances, Vol. 4, No. 1} \\ 
    https://www.ncbi.nlm.nih.gov/pmc/articles/PMC5777393/  (abgerufen am 19.03.2019)
  \item{[4]}:
    David S. Abrams, Marianne Bertrand, Sendhil Mullainathan: \textit{"Do Judges Vary in Their Treatment of Race?"},
    Publiziert 2012 in \textit{The Journal of Legal Studies, Vol. 41, No. 2} \\
    https://www.povertyactionlab.org/sites/default/files/publications/210\%20Do\%20Judges\%20Vary\%20Sept\%202010.pdf (abgerufen am 20.03.2019)
  \item{[5]}:
    Andrea Kienle, Gabriele Kunau: "\textit{Informatik und Gesellschaft - eine sozio-technische Perspektive"}, 2014
\end{itemize}

%----------------------------------------------------------------------------------------

\end{document}
